\documentclass{article}
\usepackage[utf8]{inputenc}
\usepackage[spanish]{babel}
\usepackage{listings}
\usepackage{graphicx}
\graphicspath{ {images/} }
\usepackage{cite}

\begin{document}

\begin{titlepage}
    \begin{center}
        \vspace*{1cm}
            
        \Huge
        \textbf{Primer Parcial}
            
        \vspace{0.5cm}
        \LARGE
        Informatica II
            
        \vspace{1.5cm}
            
        \textbf{Samuel Ruiz Vargas}
            
        \vfill
            
        \vspace{0.8cm}
            
        \Large
        Despartamento de Ingeniería Electrónica y Telecomunicaciones\\
        Universidad de Antioquia\\
        Medellín\\
        Marzo de 2021
            
    \end{center}
\end{titlepage}

\section{Instrucciones}
    \begin{flushleft}
    Paso 1: (todo el proceso a continuacion se hara con una sola mano) Tomar la hoja con cualquier mano y apartarla lo suficiente para que las tarjetas sean completamente libres, ahora tomar las tarjetas con la mano y levantelas dejandolas apartadas, a continuacion con la misma mano coloque la hoja en su posicion original.
    \vspace*{1cm}
    
    Paso 2: Posicionar las tarjetas encima de la hoja de papel de la misma manera que estaban al inicio, luego tomar las tarjetas y levantarlas.
    \vspace*{1cm}
    
    Paso 3: Posicionar ambas tarjetas de manera que solo uno de sus lados mas cortos toque la hoja de papel, (como las tarjetas son ambas rectangulares no importa el lado que elija mientras que sea uno de sus 2 lados mas cortos para cada tarjeta).
    \vspace*{1cm}
    
    Paso 4: Una vez colocadas, con delicadeza intentar alejar las partes bajas de las tarjetas sin dejar de tocar el papel (arrastrandolas) sin que la parte mas alta de las mismas se separen, esto con el proposito de que se soporten la una a la otra,hacer esto hasta que se cree una figura triangular, una vez se logre el equilibrio entre las 2 tarjetas dejar de tocarlas lentamente y dejarlas en esa posicion.
    \end{flushleft}
    \vspace*{5cm}

\end{document}
